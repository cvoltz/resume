% Build PDF:
%   xelatex -synctex=1 -interaction=nonstopmode -output-directory=build %.tex
% or
%   latexmk -pdf

\documentclass[10pt,letterpaper,oneside]{report}

\usepackage[singlelinecheck=false, labelfont=bf, labelsep=period]{caption}
\usepackage{changepage}
\usepackage{enumitem}
\usepackage{fancyhdr}
\usepackage{framed}
\usepackage{float}
\usepackage{fontawesome}
\usepackage{fontspec}
\usepackage[
  %showframe,
  hmargin={0.5 in, 0.5 in},
  vmargin={0.5 in, 0.75 in}
]{geometry}
\usepackage{parskip}
\usepackage[table]{xcolor} % must be before pgf since it includes xcolor
\usepackage{pgf}
\usepackage{titlesec}
\usepackage{tocloft}
\usepackage{wallpaper}
\usepackage[T1]{fontenc}
\usepackage[english]{babel}
\usepackage[weather,misc,clock]{ifsym}
\usepackage{pifont}
\usepackage{eurosym}
\usepackage{amsmath}
\usepackage{wasysym}
\usepackage{amssymb,amsfonts,textcomp}
\usepackage{color}
\usepackage{array}
\usepackage{supertabular}
\usepackage{hhline}
\usepackage{hyperref}
\hypersetup{
  colorlinks=true,
  linkcolor=blue,
  citecolor=blue,
  filecolor=blue,
  urlcolor=blue,
  pdftitle={Resume of Christopher Voltz},
  pdfauthor={Christopher Voltz},
  pdfsubject={Resume of Christopher Voltz},
  pdfkeywords={resume, Christopher Voltz}
}

% Fonts
\defaultfontfeatures{Ligatures=Tex}

% Turn off section numbering
\setcounter{secnumdepth}{0}

% List styles
% 1st level = centered square box
\renewcommand\labelitemi{\textcolor{black}{\raisebox{0.45ex}{\rule{0.6ex}{0.6ex}}}}
% 2nd level = en-dash
\renewcommand{\labelitemii}{\textendash}
% Reduce space around bullet items
\setlist{itemsep=0 ex, parsep=0 ex, partopsep=0 ex, topsep=0 ex, leftmargin=4 ex}

% Page styles
\makeatletter
\makeatother
\pagestyle{Standard}
\setlength\tabcolsep{1 mm}
\renewcommand\arraystretch{1.3}
\titleformat*{\subsection}{\large \bfseries}

% Reduce the vertical spacing around all section headers
\titlespacing{\section}
  {0 em}{1 ex plus 1 ex minus 1 ex}{-0.5 ex plus 1 ex minus 1 ex}
\titlespacing{\subsection}
  {0 em}{0 ex plus 1 ex minus 1 ex}{0.5 ex plus 1 ex minus 1 ex}
\titlespacing{\subsubsection}
  {0 em}{0 ex plus 1 ex minus 1 ex}{-1.5 ex plus 1 ex minus 1 ex}

\title{Resume of Christopher Voltz}
\author{Christopher Voltz}
\date{2018-11-05}

\begin{document}

\begin{center}
  \textbf{
    \textsc{
      Christopher Voltz\\
      6607 Evergreen Springs Court\\
      Spring, Texas 77379\\
      281-376-6816\\
      \href{mailto:voltz@voltz.ws?subject=Resume}{voltz@voltz.ws}
    }
  }
\end{center}

\section{Summary}

Experienced hardware and software engineer. Software architecture, development,
debug, test, and program management experience of desktop, embedded, and web
applications, services, drivers, and firmware for Linux, Windows, UNIX, and DOS.
In-depth knowledge of PC hardware video architecture and software APIs.
Hardware experience includes conception, architecture, design, prototyping,
debug, test, productization, and maintenance of many high speed mixed-signal
designs for video and graphics subsystems. Experience with board design, PCB
layout, agency certification, qualification, manufacturing test, vendor
evaluation/selection/management, documentation, procurement, technical support,
and legal and marketing aspects of product development. Excellent problem
solving and documentation skills.

\section{Experience}

\subsection[Hewlett Packard Enterprise]{
  \href{http://www.hpe.com/}{Hewlett Packard Enterprise}
  {\footnotesize (Hewlett-Packard, Compaq Computer Corporation)}
  \hfill November 1990--Present
}

\vspace{-1.5ex}

Currently a Master Engineer doing research and development on
{\href{https://zfsonlinux.org/}{ZFS on Linux}} and
{\href{http://lustre.org/}{Lustre}} for high performance computing (HPC) storage
solutions.

\begin{itemize}
  \item Created Ansible based installer for \href{https://ceph.com}{Ceph}
    storage solution.

  \item Lead engineer on SCRUM team developing Linux host based caching
    solution using SSDs.

  \item Developed fully automated functional test system (with web GUI) for HP
    Smart Array drivers.

  \item Developer for Open Source Linux device driver for HP Smart Array
    storage controllers for industry standard servers.

  \item Architected and developed Linux and Windows applications for desktop
    solutions (consumer and enterprise desktops, thin clients, and retail
    point of sale systems).

  \item Architected and developed education solutions (e.g., TeachNow) in
    international software development team.

  \item Member of core team which determined strategy and direction for
    commercial and enterprise desktops graphics and video.

  \item Personal Systems Group (PSG) technical focal point for graphics and
    video responsible for coordinating graphics and video technologies across
    business units within PSG and across business groups within HP.

  \item Mentor for junior engineers and interns.

  \item System administrator for Red Hat Enterprise Linux and CentOS servers
    which provide infrastructure for team.

  \item Developed multimedia applications for Linux and Windows based
    thin-clients.

  \item Lead engineer on:
    \begin {itemize}
      \item PCI Express, AGP, PCI, ISA, DVO, and sDVO embedded and add-in card
        graphics and video solutions supporting various combinations of
        DisplayPort, DVI, DFP, DMS-59, LFH-60, VGA, VAFC, VMC, Compaq Multimedia
        Bus, VFC, S-video, and composite video (input and output) connectors.

      \item Dual tuner satellite receiver board which supported conditional
        access (CA), MPEG-2, and IP data, and utilized a USB IR remote.

      \item Time-shifting standard definition tuner board and HDTV all-format
        decoder board.

      \item PC Theatre multimedia subsystem which included dual tuners, NTSC and
        PAL video decoding and de-interlacing, DVD MPEG-2 video and AC-3 audio
        decoding, high quality audio subsystem, and 3D graphics.

      \item IEEE-1394 add-in cards and ATA-100 add-in card.

      \item USB video-conferencing cameras, DV camcorders, and digital
        still-image cameras.

      \item Various video solutions utilizing analog and digital encoders and
        decoders, tuners, hardware DVD decoder, TrueQ MPEG-1 decoder option
        card, etc.

      \item Consulting engineer on standard definition (SD) and high definition
        (HD) plasma displays with integrated tuners (SD and HD), providing video
        architecture, specification, design, and testing experience.
    \end{itemize}

  \item Consulting engineer and developer for Microsoft Windows color
    management software (ColoReal).

  \item Developed software (for DOS, Windows, and Linux) for selection, test,
    and validation of graphics, video, audio, and storage systems.
    Knowledgeable with DirectX, GDI, and X Window System APIs, BIOS, driver
    development, integration, and testing, GPIB, I\textsuperscript{2}C, RS-232,
    and parallel port programming.

  \item Responsible for architecture, selection, and design of graphics, video,
    and audio subsystems on consumer and commercial desktops, and consumer
    portables. Program manager on selected projects.

  \item Specified, designed, and debugged hardware and software for DSP based
    spectrum analyzer (to test PC speakers for distortion and amplitude), analog
    data acquisition system (to test power supplies), and various other test
    fixtures to test graphics, video, and audio solutions.
\end{itemize}

\medskip

\begin{minipage}{\textwidth}
  \subsection[Texas Instruments]{
    {\href{http://www.ti.com/}{Texas Instruments} \hfill July 1989--November 1990}
  }

  Electrical Design Engineer responsible for design, debug, and documentation of
  embedded hardware and software to test production boards for
  \href{http://www.army-technology.com/projects/javelin/}{AAWS-M anti-tank
  missile (JAVELIN)}.

  \begin{itemize}
    \item Designed and debugged high speed 10-layer mixed signal PCBs for
      testing Image Array Processor board and Master Controller Processor board.

    \item Developed firmware and software using C, X Window System, and Intel
      8751 assembly language for image processing and display, and testing of
      production boards.

    \item Designed and debugged hardware and developed firmware and software for
      Bus Interface Controller Card which supported
      \href{http://www.ti.com/lit/an/ssya002c/ssya002c.pdf}{JTAG},
      \href{http://en.wikipedia.org/wiki/IEEE_488}{GPIB (IEEE-488)}, High Speed
      \href{http://www.pyramidsemiconductor.com/download/P1750A.pdf}{1750A CPU},
      and digital I/O using embedded Intel microcontroller.

    \item System administrator for HP UNIX System V workstations.
  \end{itemize}
\end{minipage}

\medskip

\begin{minipage}{\textwidth}
  \subsection[University of Dayton Research Institute]{
    \href{http://www.udri.udayton.edu/AboutUDRI/}{University of Dayton Research
    Institute} \hfill November 1986--July 1989
  }

  Computer programmer/technician responsible for design of digital image
  processing and real-time custom visual stimuli display software and analysis
  software for basic vision research using PC and SGI IRIS workstations.

  \begin{itemize}
    \item Wrote C, 80x86 assembly, FORTRAN, Pascal, and BASIC software for PCs,
      microVAXes, and CRAY supercomputer, to generate and analyze images (using
      \href{http://en.wikipedia.org/wiki/Fourier_analysis}{Fourier} and
      \href{http://en.wikipedia.org/wiki/Gabor_transform}{Gabor} analysis), and
      display images on CGA, EGA, VGA, IBM PGA, PC Vision, and DT-2871 PC option
      cards, and SGI IRIS supergraphics workstations.

    \item System administrator for UNIX System V IRIS graphics workstations.
  \end{itemize}
\end{minipage}

\section{Education}

\subsection{Continuing Education}

\begin{itemize}
  \item 2018:
    \textit{Cybersecurity for the Local Risk Environment},
    \textit{Digital Disruption in the Marketplace},
    \textit{DevOps \& Continuous Delivery within Services R\&D},
    \textit{AI/DL Software TekTalk},
    \textit{Coverity Best Practices} series,
    \textit{Gen-Z and HPC},
    \textit{LinuxKI Toolset} series,
    \textit{HPE Performance Cluster Manager}.

  \item 2016:
    \textit{Presenting Data and Information} by Edward Tufte,
    \textit{Cognitive Computing and Deep Learning in the Open Source Domain},
    \textit{A Look at Kubernetes from 30,000 Feet and a Drill Down Into the Core},
    \textit{Validation of Lustre on ZFS for the HPC Storage Market},
    \textit{Ruby Performance Optimization},
    \textit{Working with Open Source Communities},
    \textit{Lecture on OpenZFS read and write code paths} by Matt Ahrens,
    \textit{Working with the Debian Community},
    \textit{Practical Python},
    \textit{Build a High Performance Ceph Object Storage Cluster with SSD Drives
      on HP DL380 Servers},
    \textit{SUSE Linux Academy},
    \textit{Modern Agile},
    \textit{How Git Works},
    \textit{Global Trade Export Awareness},
    \textit{Haml and Sass},
    \textit{Modern Web Layout with Flexbox and CSS Grid},
    \textit{CSS Flexbox Fundamentals},
    \textit{Hands-On Ansible},
    \textit{Exploring Go: The Language Taking Over The Cloud},
    \textit{Technical Career Path (TCP)–-Overview, Nomination and Review Process},
    \textit{Intel Lustre},
    \textit{OpenShift 3: Scalable and Efficient Linux Container Platform
      Enabling DevOps with Kubernetes}.

  \item 2015:
    \textit{Linux Containers in the Enterprise with RHEL Atomic Host and Kubernetes},
    \textit{Open Source Development Framework for Web Applications},
    \textit{Just-in-Time Scan Module Calibration Using Multilinear Optimization},
    \textit{The Three Pillars Approach to Agile Testing Strategy},
    \textit{Introduction to OpenStack for HP Developers},
    \textit{Inktank Ceph Enterprise}.

  \item 2014:
    \textit{Agile Engineering at Scale},
    \textit{Object Storage 201--Understanding Architectural Trade-Offs},
    \textit{Orchestration of Digital Assets},
    \textit{Lifecycle Management for the Cloud},
    \textit{Global Trade--Technology Transfer Controls and Issues},
    \textit{Business with U.S. Public Sector Customers},
    \textit{Patent and Invention Disclosure},
    \textit{ESSN Product Development Process [PDP] Overview},
    attended Puppet Conference,
    \textit{Understanding and Maximizing Network Performance on HP ProLiant Rack
      Servers in a Linux environment},
    \textit{Shared Fault Tolerant NVM engineering seminar},
    \textit{Agile is Risk Managment},
    \textit{Customer experience and NPS},
    \textit{Ruby on Rails 4},
    \textit{C++11 Concurrency},
    \textit{Managing Large Scale Dynamic Infrastructure with Puppet},
    \textit{Cyber Defense Center: Operations, Innovation, Vision and Driving
      Product Enablement}.

  \item 2010:
    \textit{RH442VT Red Hat Enterprise System Monitoring and Performance Tuning}.

  \item 2009:
    \href{https://elearning.industriallogic.com/gh/submit?Action=AlbumContentsAction&album=collaborations&devLanguage=Cpp}
      {\textit{Microtesting Vol. 3: Collaborations}},
    \href{https://elearning.industriallogic.com/gh/submit?Action=AlbumContentsAction&album=legacy&devLanguage=Cpp}
      {\textit{Microtesting Vol. 4: Legacy Code}}.

  \item 2008:
    \textit{RHD221 Red Hat Linux Device Drivers},
    \textit{RHD236 Red Hat Linux Kernel Internals},
    \textit{Lattice FPGAs with Verilog},
    \href{https://elearning.industriallogic.com/gh/submit?Action=AlbumContentsAction&album=recognizingSmells&devLanguage=Cpp}
      {\textit{Code Smells}},
    \href{https://elearning.industriallogic.com/gh/submit?Action=AlbumContentsAction&album=foundations&devLanguage=Cpp}
      {\textit{Refactoring}},
    \href{https://elearning.industriallogic.com/gh/submit?Action=AlbumContentsAction&album=theBasics&devLanguage=Cpp}
      {\textit{Microtesting Vol. 1: The Basics}},
    \href{https://elearning.industriallogic.com/gh/submit?Action=AlbumContentsAction&album=before&devLanguage=Cpp}
      {\textit{Microtesting Vol. 2: Test-Driven Development}}.

  \item 2007: \textit{Pragmatic Studio Test-Driven Development with Rails Studio}.

  \item 2006:
    \href{http://pragmaticstudio.com/rails/}{\textit{Pragmatic Studio Ruby on Rails}},
    \href{http://pragmaticstudio.com/rails-ii}{\textit{Pragmatic Studio Advanced Rails}},
    \textit{Writing Contracts}.

  \item 2005:
    \href{http://www.sigcon.com/Pubs/HSSPsem.html}
      {\textit{Advanced High Speed Signal Propagation}},
    \textit{Accelerated Linux System Administration},
    \textit{Technical Career Path Catalyst},
    \textit{Cross-Cultural Executive Briefings},
    \textit{DHTML and CSS},
    \textit{Working with Sendmail and Apache in Linux}.

  \item 2004:
    \textit{Secure Communications and Virtual Private Networks},
    \textit{Documents on Trial: Law and Preventive Writing}.

  \item 2003:
    \textit{PCI Express},
    \textit{EMI/ESD Product Design Considerations},
    \textit{Writing UNIX Shell Programs}.

  \item 2002:
    \textit{RH253 Red Hat Linux Networking and Security Administration},
    \textit{RH300 RHCE Rapid Track Course}.

  \item 2000:
    \textit{Windows 2000 Kernel Debugging},
    \href{http://www.sigcon.com/Pubs/HSDDsem.htm}
      {\textit{High Speed Digital Design}}.
\end{itemize}

\medskip

\begin{minipage}{\textwidth}
  \subsection[MEE in Computer Systems Engineering]{
    MEE in Computer Systems Engineering at
    \href{http://www.ece.rice.edu/}{Rice University}
    {\small (Houston, TX)} \hfill May 1995
  }

  \begin{itemize}
    \item Major in VLSI design with minors in DSP/image processing and advanced
      optimizing compiler construction.

    \item Project lead and designer in a team which designed, simulated,
      fabricated and tested a full custom ASIC for vending machine controller
      using VHDL, Lager, Magic, PSPICE, and eSIM.

    \item Relevant classes include: VLSI Design I and II, Digital Signal
      Processing I, Digital Image Processing, Compiler Construction, Advanced
      Compiler Optimization, Computer Systems Design, and Digital Systems
      Design.
  \end{itemize}
\end{minipage}

\medskip

\begin{minipage}{\textwidth}
  \subsection[BSE in Computer Systems Engineering]{
    BSE in Computer Systems Engineering at
    \href{https://engineering.asu.edu/undergraduate-degree-programs}
    {Arizona State University} {\small (Tempe, AZ)} \hfill May 1989
  }

  \begin{itemize}
    \item Relevant classes include: Computer Graphics, RISC Architecture and
      Compilers, Microprogramming, Parallel Processing, Computer Architecture
      and Organization, and Operating Systems.

    \item Generated various compilers, simulators for VLIW, bit-sliced, RISC,
      CISC, and microprogrammed processors for functional and procedural
      languages.
  \end{itemize}
\end{minipage}

\section[Patents (USPTO and EPO) and Patent Applications]{
  Patents
  (\href{http://patft.uspto.gov/netacgi/nph-Parser?Sect1=PTO2&Sect2=HITOFF&p=1&u=/netahtml/PTO/search-bool.html&r=0&f=S&l=50&TERM1=voltz&FIELD1=INNM&co1=AND&TERM2=christopher&FIELD2=INNM&d=PTXT}
  {USPTO} and
  \href{http://v3.espacenet.com/searchResults?locale=en_EP&ST=quick&IA=voltz+christopher&compact=false&DB=EPODOC&submitted=true}
  {EPO})
  and Patent Applications
}

\begin{itemize}
  \item Pending: \href{http://voltz.ws/resume/20070099638.pdf}
    {\textit{Multi-number Wireless Communications System and Method}}.

  \item Pending: \href{http://voltz.ws/resume/20030123723.pdf}
    {\textit{Automatic Optimized Scanning with Color Characterization Data}}.

  \item Pending: \href{http://voltz.ws/resume/20030122840.pdf}
    {\textit{Method of Optimizing Video Output for a Computer System with
    Digital-to-Analog Converter Characterization Data}}.

  \item 8,031,268: \href{http://voltz.ws/resume/8031268.pdf}
    {\textit{Audio over a Standard Video Cable}}.

  \item 7,952,748: \href{http://voltz.ws/resume/7952748.pdf}
    {\textit{Display Device Output Adjustment System and Method}}.

  \item 7,893,998: \href{http://voltz.ws/resume/7893998.pdf}
    {\textit{Audio over a Standard Video Cable}}.

  \item 7,760,207: \href{http://voltz.ws/resume/7760207.pdf}
    {\textit{Image Display Adjustment System and Method}}.

  \item 7,609,255: \href{http://voltz.ws/resume/7609255.pdf}
    {\textit{Supplying Power from a Display Device to a Source Using a Standard
    DVI Video Cable}}.

  \item 7,398,008: \href{http://voltz.ws/resume/7398008.pdf}
    {\textit{Copy Protection for Analog Video Signals from Computing Devices}}.

  \item 7,283,430: \href{http://voltz.ws/resume/7283430.pdf}
    {\textit{Systems And Methods For Overriding An Ejection Lock}}.

  \item 6,859,538: \href{http://voltz.ws/resume/6859538.pdf}
    {\textit{Plug and play compatible speakers}}.

  \item 6,765,624: \href{http://voltz.ws/resume/6765624.pdf}
    {\textit{Simulated burst gate signal and video synchronization key for use
    in video decoding}}.

  \item 6,670,994: \href{http://voltz.ws/resume/6670994.pdf}
    {\textit{Method and apparatus for display of interlaced images on
    non-interlaced display}}.

  \item 6,504,577: \href{http://voltz.ws/resume/6504577.pdf}
    {\textit{Method and apparatus for display of interlaced images on
    non-interlaced display}}.

  \item 6,441,812: \href{http://voltz.ws/resume/6441812.pdf}
    {\textit{Hardware system for genlocking}}.

  \item 6,314,523: \href{http://voltz.ws/resume/6314523.pdf}
    {\textit{Apparatus for distributing power to a system of independently
    powered devices}}.

  \item 6,300,980: \href{http://voltz.ws/resume/6300980.pdf}
    {\textit{Computer system design for distance viewing of information and
    media and extensions to display data channel for control panel interface}}.

  \item 6,295,090: \href{http://voltz.ws/resume/6295090.pdf}
    {\textit{Apparatus for providing video resolution compensation when
    converting one video source to another video source}}.

  \item 6,201,580: \href{http://voltz.ws/resume/6201580.pdf}
    {\textit{Apparatus for supporting multiple video resources}}.

  \item 6,166,772: \href{http://voltz.ws/resume/6166772.pdf}
    {\textit{Method and apparatus for display of interlaced images on
    non-interlaced display}}.

  \item 5,892,933: \href{http://voltz.ws/resume/5892933.pdf}
    {\textit{Digital bus}}.
\end{itemize}

\section{Publications}

\begin{itemize}
  \item \href{http://voltz.ws/resume/RD519035.pdf}
    {\textit{519035 Improved RF tuning of television signals}}, Disclosed
    anonymously, HP Research Disclosure, July 2007.

  \item \href{http://voltz.ws/resume/RD519030.pdf}
    {\textit{519030 Saturation control for high definition video}}, Disclosed
    anonymously, HP Research Disclosure, July 2007.

  \item \href{http://voltz.ws/resume/RD487018.pdf}
    {\textit{487018 Hard case with embedded keyboard for a small portable
    computer}} by Christopher Voltz, HP Research Disclosure, November 2004.

  \item \href{http://www.dtic.mil/docs/citations/ADA219993}
    {\textit{Shape Discrimination Research Using an IBM PC}} by Christopher D.
    Voltz and Dr.  George A. Geri. Air Force Human Resources Laboratory Final
    Technical Report for Period Oct. 1987 to July 1989.

  \item \href{http://www.dtic.mil/docs/citations/ADA224347}
    {\textit{Texture Discrimination Research Using an IBM PC}} by Dr. George A.
    Geri and Christopher D. Voltz. Air Force Human Resources Laboratory Final
    Technical Report for Period Oct. 1987 to July 1989.
\end{itemize}

\section{Languages and Tools}

\begin{itemize}
  \item Most commonly used: Ansible, AWK, bash, C, C++, CSS, Docker, Gherkin
    (Cucumber), git, graphviz, HTML, Javascript, \LaTeX, PlantUML, Regex, RSpec,
    Ruby, sed, Sinatra, Vim.

  \item Used in previous projects: ABEL, Ada, Allegro, Assembly (1750A, 6502,
    6809, 68000, 8051, 8080, 80x86 (MASM, TASM, Intel), ADSP-21xx, Z80), AWK,
    BASIC (Fluke, Microsoft, Quick, Tiny), Chef, csh, Coverity, DOS Batch,
    dBase, FORTRAN (IV, 66, 77), Lager, Lex, MATLAB, Octave, OrCAD, Pascal
    (Turbo, UCSD), Perl, PHP, PLA, Postscript, Python, Puppet, Ruby on Rails,
    Turbo Prolog, SPICE, SQL (MySQL, SQLite), Verilog, VHDL, XML, YACC.
  \end{itemize}
\end{document}
