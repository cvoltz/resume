% Build PDF:
%   xelatex -synctex=1 -interaction=nonstopmode -output-directory=build %.tex
% or
%   latexmk -pdf

\documentclass[10pt,letterpaper,oneside]{report}

\usepackage[singlelinecheck=false, labelfont=bf, labelsep=period]{caption}
\usepackage{changepage}
\usepackage{enumitem}
\usepackage{fancyhdr}
\usepackage{framed}
\usepackage{float}
\usepackage{fontawesome}
\usepackage{fontspec}
\usepackage[
  %showframe,
  hmargin={0.5 in, 0.5 in},
  vmargin={0.5 in, 0.75 in}
]{geometry}
\usepackage{parskip}
\usepackage[table]{xcolor} % must be before pgf since it includes xcolor
\usepackage{pgf}
\usepackage{titlesec}
\usepackage{tocloft}
\usepackage{wallpaper}
\usepackage[T1]{fontenc}
\usepackage[english]{babel}
\usepackage[weather,misc,clock]{ifsym}
\usepackage{pifont}
\usepackage{eurosym}
\usepackage{amsmath}
\usepackage{wasysym}
\usepackage{amssymb,amsfonts,textcomp}
\usepackage{color}
\usepackage{array}
\usepackage{supertabular}
\usepackage{hhline}
\usepackage{hyperref}
\hypersetup{
  colorlinks=true,
  linkcolor=blue,
  citecolor=blue,
  filecolor=blue,
  urlcolor=blue,
  pdftitle={Resume of Christopher Voltz},
  pdfauthor={Christopher Voltz},
  pdfsubject={Resume of Christopher Voltz},
  pdfkeywords={resume, Christopher Voltz}
}

% Fonts
\defaultfontfeatures{Ligatures=Tex}

% Turn off section numbering
\setcounter{secnumdepth}{0}

% List styles
% 1st level = centered square box
\renewcommand\labelitemi{\textcolor{black}{\raisebox{0.45ex}{\rule{0.6ex}{0.6ex}}}}
% 2nd level = en-dash
\renewcommand{\labelitemii}{\textendash}
% Reduce space around bullet items
\setlist{itemsep=0 ex, parsep=0 ex, partopsep=0 ex, topsep=0 ex, leftmargin=4 ex}

% Page styles
\makeatletter
\makeatother
\pagestyle{Standard}
\setlength\tabcolsep{1 mm}
\renewcommand\arraystretch{1.3}
\titleformat*{\subsection}{\large \bfseries}

% Reduce the vertical spacing around all section headers
\titlespacing{\section}
  {0 em}{1 ex plus 1 ex minus 1 ex}{+0.5 ex plus 1 ex minus 1 ex}
%\titlespacing{\subsection}
%  {0 em}{0 ex plus 1 ex minus 1 ex}{0.5 ex plus 1 ex minus 1 ex}
%\titlespacing{\subsubsection}
%  {0 em}{0 ex plus 1 ex minus 1 ex}{-1.5 ex plus 1 ex minus 1 ex}

\title{Resume of Christopher Voltz}
\author{Christopher Voltz}
\date{2022-06-04}

\begin{document}

\begin{center}
  \textbf{
    \textsc{
      Christopher Voltz\\
      6607 Evergreen Springs Court\\
      Spring, Texas 77379\\
      281-376-6816\\
      \href{mailto:voltz@voltz.us?subject=Resume}{voltz@voltz.us}
    }
  }
\end{center}

\section{Summary}

Experienced software and hardware engineer. Software architecture, development,
debug, test, and program management experience of desktop, embedded, and web
applications, services, drivers, and firmware for Linux, Windows, UNIX, and DOS.
In-depth knowledge of HPC storage and PC graphics and video. Hardware
experience includes conception, architecture, design, prototyping, debug, test,
productization, and maintenance of many high speed mixed-signal designs for
video and graphics subsystems. Experience with board design, PCB layout, agency
certification, qualification, manufacturing test, vendor
evaluation/selection/management, documentation, procurement, technical support,
and legal and marketing aspects of product development. Excellent problem
solving and documentation skills.

\section{Experience}

\subsection[Hewlett Packard Enterprise]{
  \href{http://www.hpe.com/}{Hewlett Packard Enterprise}
  {\footnotesize (Hewlett-Packard, Compaq Computer Corporation)}
  \hfill November 1990--Present
}

\vspace{-1.5ex}

Currently a Master Engineer who is the software architect and technical team
lead for an international team doing research and development of
{\href{https://www.hpe.com/psnow/doc/a50002561enw.pdf?jumpid=in_pdp-psnow-qs}{HPE
Parallel File System Storage}} for HPC and enclosure management utilities using
Scrumban.

\begin{itemize}
  \item Developed software for {\href{https://zfsonlinux.org/}{ZFS on Linux}}
    and {\href{http://lustre.org/}{Lustre}} for HPC storage solutions.

  \item Created Ansible based installer for \href{https://ceph.com}{Ceph}
    storage solution.

  \item Lead engineer on SCRUM team developing Linux host based caching
    solution using SSDs.

  \item Developed fully automated functional test system (with web GUI) for HP
    Smart Array drivers.

  \item Developer for Open Source Linux device driver for HP Smart Array
    storage controllers for industry standard servers.

  \item Architected and developed Linux and Windows applications for desktop
    solutions (consumer and enterprise desktops, thin clients, and retail
    point of sale systems).

  \item Architected and developed education solutions (e.g., TeachNow) in
    international software development team.

  \item Member of core team which determined strategy and direction for
    commercial and enterprise desktops graphics and video.

  \item Technical focal point for graphics and
    video responsible for coordinating graphics and video technologies across
    business groups within HP.

  \item Mentor for junior engineers and interns.

  \item System administrator for Red Hat Enterprise Linux and CentOS servers
    which provide infrastructure for team.

  \item Developed multimedia applications for Linux and Windows based
    thin-clients.

  \item Lead engineer on:
    \begin {itemize}
      \item PCI Express, AGP, PCI, ISA, DVO, and sDVO embedded and add-in card
        graphics and video solutions supporting various combinations of
        DisplayPort, DVI, DFP, DMS-59, LFH-60, VGA, VAFC, VMC, Compaq Multimedia
        Bus, VFC, S-video, and composite video (input and output) connectors.

      \item Dual tuner satellite receiver board which supported conditional
        access (CA), MPEG-2, and IP data, and utilized a USB IR remote.

      \item Time-shifting standard definition tuner board and HDTV all-format
        decoder board.

      \item PC Theatre multimedia subsystem which included dual tuners, NTSC and
        PAL video decoding and de-interlacing, DVD MPEG-2 video and AC-3 audio
        decoding, high quality audio subsystem, and 3D graphics.

      \item IEEE-1394 add-in cards and ATA-100 add-in card.

      \item USB video-conferencing cameras, DV camcorders, and digital
        still-image cameras.

      \item Various video solutions utilizing analog and digital encoders and
        decoders, tuners, hardware DVD decoder, TrueQ MPEG-1 decoder option
        card, etc.

      \item Consulting engineer on standard definition (SD) and high definition
        (HD) plasma displays with integrated tuners (SD and HD), providing video
        architecture, specification, design, and testing experience.
    \end{itemize}

  \item Consulting engineer and developer for Microsoft Windows color
    management software (ColoReal).

  \item Developed software (for DOS, Windows, and Linux) for selection, test,
    and validation of graphics, video, audio, and storage systems.
    Knowledgeable with DirectX, GDI, and X Window System APIs, BIOS, driver
    development, integration, and testing, GPIB, I\textsuperscript{2}C, RS-232,
    and parallel port programming.

  \item Responsible for architecture, selection, and design of graphics, video,
    and audio subsystems on consumer and commercial desktops, and consumer
    portables. Program manager on selected projects.

  \item Specified, designed, and debugged hardware and software for DSP based
    spectrum analyzer (to test PC speakers for distortion and amplitude), analog
    data acquisition system (to test power supplies), and various other test
    fixtures to test graphics, video, and audio solutions.
\end{itemize}

\medskip

\begin{minipage}{\textwidth}
  \subsection[Texas Instruments]{
    {\href{http://www.ti.com/}{Texas Instruments} \hfill July 1989--November 1990}
  }

  Electrical Design Engineer responsible for design, debug, and documentation of
  embedded hardware (including high speed 10-layer mixed signal PCBs) and
  software using C, X Window System, and Intel 8751 assembly language.
\end{minipage}

\medskip

\begin{minipage}{\textwidth}
  \subsection[University of Dayton Research Institute]{
    \href{http://www.udri.udayton.edu/AboutUDRI/}{University of Dayton Research
    Institute} \hfill November 1986--July 1989
  }

  Computer programmer/technician responsible for design of digital image
  processing and real-time custom visual stimuli display and analysis
  software (using C, 80x86 assembly, FORTRAN, Pascal, and BASIC) for basic
  vision research.
\end{minipage}

\section{Education}

\begin{minipage}{\textwidth}
  \subsection[MEE in Computer Systems Engineering]{
    MEE in Computer Systems Engineering at
    \href{http://www.ece.rice.edu/}{Rice University}
    {\small (Houston, TX)} \hfill May 1995
  }

  \begin{itemize}
    \item Major in VLSI design with minors in DSP/image processing and advanced
      optimizing compiler construction.
  \end{itemize}
\end{minipage}

\medskip

\begin{minipage}{\textwidth}
  \subsection[BSE in Computer Systems Engineering]{
    BSE in Computer Systems Engineering at
    \href{https://engineering.asu.edu/undergraduate-degree-programs}
    {Arizona State University} {\small (Tempe, AZ)} \hfill May 1989
  }
\end{minipage}

\section[Patents (USPTO and EPO) and Patent Applications]{
  Patents
  (\href{http://patft.uspto.gov/netacgi/nph-Parser?Sect1=PTO2&Sect2=HITOFF&p=1&u=/netahtml/PTO/search-bool.html&r=0&f=S&l=50&TERM1=voltz&FIELD1=INNM&co1=AND&TERM2=christopher&FIELD2=INNM&d=PTXT}
  {USPTO} and
  \href{http://v3.espacenet.com/searchResults?locale=en_EP&ST=quick&IA=voltz+christopher&compact=false&DB=EPODOC&submitted=true}
  {EPO})
  and Patent Applications
}

\begin{itemize}
  \item Pending: \href{http://voltz.us/resume/20070099638.pdf}
    {\textit{Multi-number Wireless Communications System and Method}}.

  \item Pending: \href{http://voltz.us/resume/20030123723.pdf}
    {\textit{Automatic Optimized Scanning with Color Characterization Data}}.

  \item Pending: \href{http://voltz.us/resume/20030122840.pdf}
    {\textit{Method of Optimizing Video Output for a Computer System with
    Digital-to-Analog Converter Characterization Data}}.

  \item 11,340,889 \href{http://voltz.us/resume/11_340_889.pdf}
    {\textit{Updating Firmware Images on Chained Input/Output (I/O) Modules}}.

  \item 8,358,347 \href{http://voltz.us/resume/8358347.pdf}
    {\textit{Frame rate measurement}}.

  \item 8,031,268: \href{http://voltz.us/resume/8031268.pdf}
    {\textit{Audio over a Standard Video Cable}}.

  \item 7,952,748: \href{http://voltz.us/resume/7952748.pdf}
    {\textit{Display Device Output Adjustment System and Method}}.

  \item 7,893,998: \href{http://voltz.us/resume/7893998.pdf}
    {\textit{Audio over a Standard Video Cable}}.

  \item 7,760,207: \href{http://voltz.us/resume/7760207.pdf}
    {\textit{Image Display Adjustment System and Method}}.

  \item 7,609,255: \href{http://voltz.us/resume/7609255.pdf}
    {\textit{Supplying Power from a Display Device to a Source Using a Standard
    DVI Video Cable}}.

  \item 7,398,008: \href{http://voltz.us/resume/7398008.pdf}
    {\textit{Copy Protection for Analog Video Signals from Computing Devices}}.

  \item 7,283,430: \href{http://voltz.us/resume/7283430.pdf}
    {\textit{Systems And Methods For Overriding An Ejection Lock}}.

  \item 6,859,538: \href{http://voltz.us/resume/6859538.pdf}
    {\textit{Plug and play compatible speakers}}.

  \item 6,765,624: \href{http://voltz.us/resume/6765624.pdf}
    {\textit{Simulated burst gate signal and video synchronization key for use
    in video decoding}}.

  \item 6,670,994: \href{http://voltz.us/resume/6670994.pdf}
    {\textit{Method and apparatus for display of interlaced images on
    non-interlaced display}}.

  \item 6,504,577: \href{http://voltz.us/resume/6504577.pdf}
    {\textit{Method and apparatus for display of interlaced images on
    non-interlaced display}}.

  \item 6,441,812: \href{http://voltz.us/resume/6441812.pdf}
    {\textit{Hardware system for genlocking}}.

  \item 6,314,523: \href{http://voltz.us/resume/6314523.pdf}
    {\textit{Apparatus for distributing power to a system of independently
    powered devices}}.

  \item 6,300,980: \href{http://voltz.us/resume/6300980.pdf}
    {\textit{Computer system design for distance viewing of information and
    media and extensions to display data channel for control panel interface}}.

  \item 6,295,090: \href{http://voltz.us/resume/6295090.pdf}
    {\textit{Apparatus for providing video resolution compensation when
    converting one video source to another video source}}.

  \item 6,201,580: \href{http://voltz.us/resume/6201580.pdf}
    {\textit{Apparatus for supporting multiple video resources}}.

  \item 6,166,772: \href{http://voltz.us/resume/6166772.pdf}
    {\textit{Method and apparatus for display of interlaced images on
    non-interlaced display}}.

  \item 5,892,933: \href{http://voltz.us/resume/5892933.pdf}
    {\textit{Digital bus}}.
\end{itemize}

\section{Publications}

\begin{itemize}
  \item \href{http://voltz.us/resume/RD519035.pdf}
    {\textit{519035 Improved RF tuning of television signals}}, Disclosed
    anonymously, HP Research Disclosure, July 2007.

  \item \href{http://voltz.us/resume/RD519030.pdf}
    {\textit{519030 Saturation control for high definition video}}, Disclosed
    anonymously, HP Research Disclosure, July 2007.

  \item \href{http://voltz.us/resume/RD487018.pdf}
    {\textit{487018 Hard case with embedded keyboard for a small portable
    computer}} by Christopher Voltz, HP Research Disclosure, November 2004.

  \item \href{http://voltz.us/resume/ADA219993}
    {\textit{Shape Discrimination Research Using an IBM PC}} by Christopher D.
    Voltz and Dr.  George A. Geri. Air Force Human Resources Laboratory Final
    Technical Report for Period Oct. 1987 to July 1989.

  \item \href{http://voltz.us/resume/ADA224347}
    {\textit{Texture Discrimination Research Using an IBM PC}} by Dr. George A.
    Geri and Christopher D. Voltz. Air Force Human Resources Laboratory Final
    Technical Report for Period Oct. 1987 to July 1989.
\end{itemize}

\end{document}
